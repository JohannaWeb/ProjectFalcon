\documentclass[11pt,a4paper]{article}
\usepackage[utf8]{inputenc}
\usepackage[T1]{fontenc}
\usepackage{amsmath, amssymb, amsthm}
\usepackage{hyperref}
\usepackage{graphicx}
\usepackage{geometry}
\geometry{margin=1in}

\title{\textbf{Project Falcon: An Adversarial Algorithmic Trust Protocol for Decentralized Social Identity}}
\author{Johanna Almeida \\ \texttt{johanna@falcon-project.io} \\ \textit{Project Falcon Maintainer}}
\date{February 2026}

\begin{document}

\maketitle

\begin{abstract}
Current digital reputation systems are centrally managed, gammable, and fail to provide high-integrity verification in adversarial environments. Project Falcon introduces an automated, zero-trust reputation framework designed for the AT Protocol ecosystem. By utilizing subjective transitive trust with temporal decay and non-linear squashing, Falcon establishes a mathematical standard for decentralized trust that is resistant to Sybil attacks and collusion. This paper outlines the architectural implementation, the mathematical foundations of the scoring engine, and the protocol's integration with on-chain attestation services.
\end{abstract}

\section{Introduction}
As the digital landscape transitions toward decentralized protocols like AT Protocol, the fundamental problem of identity shifts from "who are you" to "why should I trust you." Centralized platforms have historically monetized this problem through opaque algorithms. Project Falcon flips this paradigm by making trust an undeniable, mathematically-derived protocol utility.

\section{System Architecture}
Project Falcon is built as a sovereign microservices suite designed for high-concurrency and cryptographic rigor.

\subsection{The Edge Gateway}
The entry point of the protocol implements a zero-trust model. Every request is verified against the AT Protocol DID (Decentralized Identifier) document. A custom Secp256k1/P-256 JWT validation engine ensures that identities are cryptographically proven before a reputation score is ever calculated.

\subsection{Sovereign Integration Vessels (SIV)}
Falcon merges social identity with work identity through SIVs. These high-throughput workers, powered by Java 21+ Virtual Threads, ingest real-time activity from platforms like GitHub, Jira, and Linear, providing an empirical baseline for the trust engine.

\section{Mathematical Implementation}
% Include the math section here
Project Falcon defines trust as a subjective, transitive, and decayable resource. The trust score $T$ between a viewer $u$ and a target $v$ is calculated using a dynamic aggregation of direct observations and transitive bridges.

\subsection{The Temporal Decay Function}
Trust is not a static attribute; its validity decreases as the observation age increases. We define the temporal weight $W_t$ of a relation established at time $t_{rel}$ as:

\[ W_t(\Delta t) = e^{-\lambda \Delta t} \]

where $\Delta t = t_{now} - t_{rel}$ and $\lambda$ is the decay constant. This ensure that stale identities naturally revert to a neutral state without continuous verification.

\subsection{Subjective Transitive Trust (STT)}
For a target $v$ with no direct relation to $u$, the score is derived through a set of intermediate bridges $w \in B(u, v)$. The total score $S(u, v)$ is defined as:

\[ S(u, v) = \tanh \left( \frac{\sum_{w \in B(u, v)} T(u, w) \cdot T(w, v) \cdot W_t(\Delta t_w)}{\sum_{w \in B(u, v)} |T(u, w)|} \right) \]

where:
\begin{itemize}
    \item $T(u, w)$ is the viewer's trust in the bridge (the "subjective weight").
    \item $T(w, v)$ is the bridge's observation of the target.
    \item $\tanh$ is the non-linear squashing function used for Sybil resistance.
\end{itemize}

\subsection{Adversarial Resistance}
To prevent "Trust Farming" collateral, the protocol implements a non-linear penalty for conflicting bridges. Significant variance in bridge reports causes the cumulative score to normalize toward zero, flagging the target for manual review.


\section{Security & Adversarial Resistance}

Falcon is architected for an environment where participants are assumed to be rational and potentially adversarial agents. The protocol employs three primary mechanisms to ensure the integrity of the trust score $S(u, v)$ in the face of coordinated attacks.

\subsection{Non-Linear Sybil Resistance}
The most prevalent threat to decentralized reputation is the Sybil attack, where an adversary creates a large number of identities to inflate a target's score. Falcon mitigates this through the non-linear squashing function $\sigma(x) = \tanh(\kappa x)$. 

As the aggregate trust from intermediate bridges $w$ increases, the marginal gain for $S(u, v)$ follows a diminishing returns curve. An adversary attempting to "farm" trust must exponentially increase the number of high-reputation bridges to achieve a linear increase in the target's final score, significantly raising the economic cost of the attack.

\subsection{Transitive Integrity and Entropy Penalties}
Unlike static web-of-trust models, Falcon analyzes the \textit{consistency} of bridge reports. If a set of bridges $B(u, v)$ provides highly divergent scores for a target $v$, the system identifies a "Trust Conflict." 

We define the conflict coefficient $C$ as the normalized variance of bridge reports. When $C$ exceeds a threshold $\theta$, the protocol applies an entropy penalty:
\[ S_{final} = S(u, v) \cdot (1 - C) \]
This ensures that "Controversial" entities—those who are trusted by one cluster but distrusted by another—cannot achieve a "Perfectly Trusted" status, regardless of the sheer number of trusting bridges.

\subsection{Proof of Continuous Integrity (Temporal Decay)}
Temporal decay ($e^{-\lambda \Delta t}$) serves as a dynamic security layer. It forces an adversary to maintain a "Proof of Activity." To keep a fraudulent reputation score high, an attacker must continuously forge high-integrity interactions across their bridge network. Stale relations naturally revert to a neutral state ($0.0$), ensuring that the system's "memory" is prioritized toward recent, verified cryptographic proof.

\subsection{Cryptographic Hardening at the Edge}
To prevent DID-spoofing and token replay attacks, the Falcon Gateway enforces strict ECDSA verification (Secp256k1/P-256) at the entry point. By resolving the DID document and verifying the JWT signature against the raw public key coordinates found in the user’s sovereign record, Falcon ensures that every reputation update is backed by an authentic, cryptographically-proven identity.


\section{On-Chain Attestation}
To ensure interoperability, Falcon scores are registered on the Ethereum Attestation Service (EAS). This allows external protocols to "gate" access based on Falcon reputation scores without needing direct access to the Falcon backend.

\section{Conclusion}
Project Falcon provides a robust, mathematically rigorous framework for solving the decentralized reputation problem. By moving trust from the realm of corporate heuristic to algorithmic standard, we enable a truly sovereign social layer for the internet.

\bibliographystyle{plain}
\bibliography{refs}

\end{document}
